%!TEX root = ../thesis.tex
%*******************************************************************************
%*********************************** First Chapter *****************************
%*******************************************************************************

\chapter{Introduction}  %Title of the First Chapter
\label{chapter1}

\ifpdf
\graphicspath{{Chapter1/Figs/Raster/}{Chapter1/Figs/PDF/}{Chapter1/Figs/}}
\else
\graphicspath{{Chapter1/Figs/Vector/}{Chapter1/Figs/}}
\fi

\section{Thesis Contributions}
\label{sec:contributions}

\section{Thesis Structure}
\label{sec:structure}

%********************************** %Chapter 1 Nomenclature **************************************

% Other symbols
\nomenclature[x-expectation]{$\mathbb{E}$}{Expectation}
\nomenclature[x-variance]{$\operatorname{Var}$}{Variance}

% Machine Learning Shorthands
\nomenclature[Z-ml]{ML}{Machine Learning}
\nomenclature[Z-dl]{DL}{Deep Learning}
\nomenclature[Z-vae]{VAE}{Variational Autoencoder}
\nomenclature[Z-gan]{GAN}{General Adversarial Network}
\nomenclature[Z-nn]{NN}{Neural Network}
\nomenclature[Z-cnn]{CNN}{Convolutional Neural Network}

% Monte Carlo Shorthands
\nomenclature[Z-qmc]{QMC}{Quantum Monte Carlo}
\nomenclature[Z-vmc]{VMC}{Variational Quantum Monte Carlo}
\nomenclature[Z-dmc]{DMC}{Diffusion Quantum Monte Carlo}

% Latin expressions and shorthands
\nomenclature[z-pbc]{p.b.c}{Periodic boundary condition}
\nomenclature[z-eg]{e.g.}{Exempli gratia ("for the sake of an example")}
\nomenclature[z-ie]{i.e.}{Id est ("it is")}
\nomenclature[z-iid]{i.i.d}{Independent and identically distributed}
\nomenclature[z-st]{s.t.}{Such that}
\nomenclature[z-wrt]{w.r.t}{With respect to}