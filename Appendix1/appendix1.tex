%!TEX root = ../thesis.tex
% ******************************* Thesis Appendix A ****************************

\chapter{Additional Derivations}

\section{Holland cost}
\label{app:holland-cost}
Following Holland's~\cite{holland1977cost} derivation of a stochastic control problem for the Schr\" odinger equation. We start by introducing the exponential transform
\begin{equation}
	\psi(x)=\exp [-U(x)],
\end{equation}
into the Schro\" odinger equation for $x \in \mathbb{R}^{n}$
\begin{equation}
	H \psi=\left[-\frac{1}{2} \nabla^{2}+V(x)\right] \psi=\lambda \psi,
\end{equation}
to obtain
\begin{equation}
	\label{eq:a-hc1}
	\frac{1}{2} \nabla^{2} U-\frac{1}{2}(\nabla U)^{2}+V(x)=\lambda.
\end{equation}
We can reinterpret the second term in eq.~\eqref{eq:a-hc1} as a minimum over all vectors $v$ for every $x \in \mathbb{R}^{n}$
\begin{equation}
	\label{eq:a-hc2}
	\frac{1}{2} \nabla^{2} U+\min_{v}\left[v \cdot \nabla U+\frac{1}{2}|v|^{2}+V(x)\right]=\lambda.
\end{equation}
We define $v$ to be a Lipshitz continuous function, i.e. the drift $v(x)$. Each drift now generates an It\^ o process
\begin{equation}
	\label{eq:a-hc3}
	\mathrm{d}X_{t}=\mathrm{d}W_{t}+v\left(X_{t}\right) \mathrm{d}t; \quad \text{ and } \quad X_0 = x,
\end{equation}
and we can define the cost function $C[v]$ for each $v$ as
\begin{equation}
	\label{eq:a-hc4}
	C[v]=\lim _{T \rightarrow \infty} \frac{1}{T} \mathbb{E}\left[\int \left(\frac{1}{2}\left|v\left(X_{t}\right)\right|^{2}+V\left(X_{t}\right)\right)\mathrm{d}x \right].
\end{equation}
Holland~\cite{holland1977cost} proves the following theorem.
\begin{theorem}[Holland cost function]
	The minimum $\lambda = \min_{v} C[v]$, where the minimum is taken over drift functions $v: \Omega \rightarrow \mathbb{R}^n$, $\Omega \subset \mathbb{R}^n$ with Neumann boundary conditions $\frac{\partial \psi}{\partial n}=0 \text{ on } \partial \Omega$, is obtained only for $v=\frac{\nabla \psi}{\psi}=-\nabla U$.
\end{theorem}
\begin{proof}
	From eq.~\eqref{eq:a-hc2} follows the inequality
	\begin{equation}
		\label{eq:a-hc5}
		\frac{1}{2} \nabla^{2} U+v \cdot \nabla U+\frac{1}{2}|v(x)|^{2}+V(x) \geq \lambda,
	\end{equation}
	we notice that the first two terms are an infinitesimal generator $[\mathcal{L}_G U](x)$ of the process in eq.~\eqref{eq:a-hc3}, which is defined as
	\begin{equation}
		\left[\mathcal{L}_{\mathrm{G}} \psi\right](x) \equiv \lim _{t \rightarrow 0} \frac{\mathbb{E}_{x}\left[\psi\left(X_{t}\right)\right]-\psi(x)}{t},
	\end{equation} 
	where $\mathbb{E}_x$ is over processes with initial condition $X_0 = x$.
	With this in mind, we evaluate eq.~\eqref{eq:a-hc5} on the process $X_t$ and integrate over time to obtain
	\begin{equation}
		\frac{\mathbb{E}\left[U\left(X_{T}\right)-U(x)\right]}{T}+\mathbb{E}\left[\frac{1}{T}\int_{0}^{T} \left(\frac{1}{2}\left|v\left(X_{t}\right)\right|^{2}+V\left(X_{t}\right)\right)\mathrm dt\right] \geq \lambda.
	\end{equation}
	In the $T \rightarrow \infty$ limit, the first terms disappears, so long as $U$ is bounded, and we are left with $C[v]$, meaning that we have proven the bound 
	\begin{equation}
		C[v] \geq \lambda.
	\end{equation}
	We see that the minimum $\lambda$ is achieved for control function
	\begin{equation}
		v = \frac{\nabla \psi}{\psi} = -\nabla U,
	\end{equation}
	in eq.~\eqref{eq:a-hc2}, the uniqueness of this optimal $v$ is due to the fact that if $v \neq \frac{\nabla \psi}{\psi}$ there cannot be a full equality in eq.~\eqref{eq:a-hc5}.
\end{proof}


\section{Probabilistic interpretation of the Holland cost function}
\label{app:holland-prob}
This section follows Barr et al.~\cite{barr2020quantum} which is closely related to previous work~\cite{dai1990markov}. To express the RN derivative we use the Girsanov theorem for both $\mathbb{P}_v$
\begin{equation}
	\frac{\mathrm{d} \mathbb{P}_{v}}{\mathrm{d} \mathbb{P}_{0}}=\exp \left(\int v\left(X_{t}\right) \mathrm{d} X_{t}-\frac{1}{2} \int\left|v\left(X_{t}\right)\right|^{2} \mathrm{d} t\right),
\end{equation}
and $\mathbb{P}_{\mathrm{FK}}$
\begin{equation}
	\label{eq:a-hp2}
	\frac{\mathrm{d} \mathbb{P}_{\mathrm{FK}}}{\mathrm{d} \mathbb{P}_{0}}=\mathcal{N} \exp \left(-\int V\left(X_{t}\right) \mathrm{d} t\right).
\end{equation}
We combine both, and we have up to exponential accuracy $\mathcal{N} \sim e^{\lambda T}$
\begin{equation}
	\log \left(\frac{\mathrm{d} \mathbb{P}_{v}}{\mathrm{d} \mathbb{P}_{\mathrm{FK}}}\right)=\log \left(\frac{\mathrm{d} \mathbb{P}_{v}}{\mathrm{d} \mathbb{P}_{0}} \frac{\mathrm{d} \mathbb{P}_{0}}{\mathrm{d} \mathbb{P}_{\mathrm{FK}}}\right) = \int v\left(X_{t}\right) \mathrm{d} X_{t}+\int \left(-\frac{1}{2}\left|v\left(X_{t}\right)\right|^{2}+V\left(X_{t}\right)\right)\mathrm{d} t - \lambda T,
\end{equation}
finally we substitute $\mathrm{d} X_t = \mathrm{d}W_t + v(X_t)\mathrm{d}t$ to get
\begin{equation}
	\begin{aligned}
	\log \left(\frac{\mathrm{d} \mathbb{P}_{v}}{\mathrm{d} \mathbb{P}_{\mathrm{FK}}}\right) 
	& =  \int v\left(X_{t}\right) \mathrm{d} W_{t} + \int |v(X_t)|^2 \mathrm{d}t + \int \left(-\frac{1}{2}\left|v\left(X_{t}\right)\right|^{2}+V\left(X_{t}\right)\right)\mathrm{d} t\\
	& =  \int v\left(X_{t}\right) d W_{t}+\int \left(\frac{1}{2}\left|v\left(X_{t}\right)\right|^{2}+V\left(X_{t}\right)\right) \mathrm{d}t.
	\end{aligned}
\end{equation}
The $D_{KL}$ and its long-time limit follow from the logarithm Radon-Nikodym derivative
\begin{equation}
	D_{\mathrm{KL}}\left(\mathbb{P}_{v} \| \mathbb{P}_{\mathrm{FK}}\right) 
	\underset{T \rightarrow \infty}{\longrightarrow} \underset{\mathbb{P}_{v}}{\mathbb{E}}\left[\int  \left(\frac{1}{2}\left|v\left(X_{t}\right)\right|^{2}+V\left(X_{t}\right)\right)\mathrm{d}t \right]-\lambda T.
\end{equation}
A closer look at the normalisation constant $\mathcal{N}$ in eq.~\eqref{eq:a-hp2} gives rise to the boundary term. The normalisation is given by
\begin{equation}
	\mathcal N = \frac{\tilde{\psi}\left(\boldsymbol{r}_{T}, T\right)}{\bar{\psi}\left(\boldsymbol{r}_{0}, 0\right)},
\end{equation}
where $\tilde{\psi}\left(r_{t}, t\right)$ is the solution to the backwards imaginary time Schr\" odinger equation~\cite{barr2020quantum}, and is related to the distribution of the stochastic process in eq.~\eqref{eq:a-hc3} as
\begin{equation}
	\pi(r, t) = \tilde{\psi}\left(r, t\right) \psi\left(r, t\right).
\end{equation}
If we now take both distributions at initial time $\pi(r, 0)$ and terminal time $\pi(r, T)$ to be the ground state the normalisation constant becomes
\begin{equation}
	\frac{\tilde{\psi}\left(\boldsymbol{r}_{T}, T\right)}{\bar{\psi}\left(\boldsymbol{r}_{0}, 0\right)} = e^{E_{0} T} \frac{\varphi_{0}\left(\boldsymbol{r}_{T}\right)}{\varphi_{0}\left(\boldsymbol{r}_{0}\right)},
\end{equation}
and accounts for the boundary term and $E_0 T$ term in the Kullback-Leibler divergence.


\newpage 
\section{Todorov cost}

\section{Probabilistic interpretation of the Todorov cost function}