%!TEX root = ../thesis.tex
% **************************** Define Graphics Path **************************
\ifpdf
\graphicspath{{Chapter5/Figs/Raster/}{Chapter5/Figs/PDF/}{Chapter5/Figs/}}
\else
\graphicspath{{Chapter5/Figs/Vector/}{Chapter5/Figs/}}
\fi

%*******************************************************************************
%****************************** Fifth Chapter *********************************
%*******************************************************************************

\chapter{Experiments and Results}
\label{chapter5}
\hl{todo}
\begin{enumerate}
	\item training experiments
	\item sampling experiments
\end{enumerate}

\section{Learning the rates} 
\begin{figure}
	\centering
	\includegraphics[width=\linewidth]{Chapter5/Figs/Vector/init_test_learning}
	\caption[Initial rate training experiments]{\textbf{Initial rate training experiments}}
	\label{fig:inittestlearning}
\end{figure}

\begin{figure}
	\centering
	\includegraphics[width=\linewidth]{Chapter5/Figs/Raster/avg_var_loss}
	\caption[Performance of the pCNN for different setups]{\textbf{Performance of the pCNN for different setups.}}
	\label{fig:avgvarloss}
\end{figure}

\begin{figure}
	\centering
	\includegraphics[width=\linewidth]{Chapter5/Figs/Raster/initial_time_space}
	\caption[Time and space complexity of the pCNN]{\textbf{Time and space complexity of the pCNN}}
	\label{fig:initialtimespace}
\end{figure}

%\subsubsection{How does simulation time $T$ and batch size $N_b$ affect the learning of the rates?}
%
%\subsubsection{How do architectural choices of the pCNN limit learning the rates?}
%
%\subsubsection{How does the size of lattice play a role?}
%
%\subsubsection{Can we improve learning by scaling the output of the pCNN?}
%Also mention reshuffling.
%
%\subsubsection{What role does symmetry play in all of this?}

\section{Importance sampling}
\section{Results}
\subsection{Ising model}
\subsubsection{One dimension}
\begin{figure}[H]
	\centering
	\subfloat[$E_0/N$]{\includegraphics[width=0.345\linewidth]{Chapter5/Figs/Vector/tfim1d_en_finite_scaling.pdf}}
	\subfloat[$\sigma_z$]{\includegraphics[width=0.33\linewidth]{Chapter5/Figs/Vector/tfim1d_mz_finite_scaling.pdf}}
	\subfloat[$\sigma_x$]{\includegraphics[width=0.33\linewidth]{Chapter5/Figs/Vector/tfim1d_mx_finite_scaling.pdf}}
	\caption[TFIM-1d energy and magnetisation]{\textbf{TFIM-1d energy and magnetisation}}
	\label{fig:tfim1d_finite_scaling_properties}
\end{figure}

\subsection{XY model}


%\begin{figure}[H]
%	\centering
%	\includegraphics[width=\linewidth]{Chapter6/Figs/Vector/bvl_single}
%	\caption[Effect of batch size $N_b$ on training with the endpoint loss.]{\textbf{Effect of batch size $N_b$ on training with the endpoint loss.}}
%	\label{fig:bvlsingle}
%\end{figure}
%\begin{figure}[H]
%	\centering
%	\includegraphics[width=\linewidth]{Chapter6/Figs/Vector/tvl_single}
%	\caption[Effect of simulation time $T$ on training with the endpoint loss.]{\textbf{Effect of simulation time $T$ on training with the endpoint loss.}}
%	\label{fig:tvlsingle}
%\end{figure}
%\begin{figure}[H]
%	\centering
%	\includegraphics[width=\linewidth]{Chapter6/Figs/Vector/wvl_single}
%	\caption[Effect of pCNN width $k$ on training with the endpoint loss.]{\textbf{Effect of pCNN width $k$ on training with the endpoint loss.}}
%	\label{fig:wvlsingle}
%\end{figure}
%\begin{figure}[H]
%	\centering
%	\includegraphics[width=\linewidth]{Chapter6/Figs/Vector/lvl_single}
%	\caption[Effect of pCNN depth $l$ on training with the endpoint loss.]{\textbf{Effect of pCNN depth $l$ on training with the endpoint loss.}}
%	\label{fig:lvlsingle}
%\end{figure}