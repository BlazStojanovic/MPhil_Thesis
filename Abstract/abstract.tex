% ************************** Thesis Abstract *****************************
% Use `abstract' as an option in the document class to print only the titlepage and the abstract.
\begin{abstract}


The Feynman-Kac formula provides a bridge between stochastic processes and partial differential equations. It can be used to formulate the ground state problem of a quantum many-body system in terms of evaluating an expectation with respect to the Feynman-Kac path measure. In this work we make use of connections between optimal control, quantum mechanics and stochastic processes to propose a method that obtains the Feynman-Kac measure for stoquastic Hamiltonians. The trajectory distribution of a quantum lattice model is represented with transition rates of a continuous time Markov chain, which are learned by minimising the $\log \text{RN}$ loss with fixed endpoints. Finding the optimal rates is equivalent to knowing the Feynman-Kac path measure and permits optimal importance sampling from the ground state of the model. The method is implemented and demonstrated on the transverse field Ising model. 

\vspace{3em}
\noindent

Word count: $14,998$

\end{abstract}
